%chapter introduce un nuevo cap�tulo
\chapter{Resumen}

Este proyecto se resume en un estudio de como aprovechar las tecnolog�as que durante estos �ltimos a�os han revolucionado el mundo de la web (Realtime con Sockets.IO, Node.JS para aprovechar el motor V8 de Google, MongoDB, Ionic para el uso de Apps H�bridas...) para darle una aplicaci�n al mundo de la dom�tica.

\vspace{5mm}

En primer lugar, se plantearan una serie de tecnolog�as muy ligadas al �mbito del desarrollo web y como estas, ser�n orientadas hacia una aplicaci�n sencilla de dom�tica, la cual ser� explicada en el apartado [PONER APARTADO].

\vspace{5mm}

En segundo lugar, se les ser� explicado como estar� estructurada la aplicaci�n, es decir, materiales, alcanze, objetivos y finalidad. En lineas generales, el objetivo consistir� en realizar un sistema dom�tico a escala reducida con sensores de proximidad, temperatura, conexi�n de alarma y sistemas de iluminaci�n. Todo ello monotorizado para que se pueda controloar via m�vil o via ordenador. Actualmente y como les sera explicado en el apartado [Citar luego], existen varios sistemas para controlar a peque�a escala el problema del que hablamos. Pero sin embargo, estas aplicaciones solo son operativas en un entorno local, por lo que el problema al que este trabajo alude tratara de dar llevar un entorno local (No el de esas aplicaciones, sino uno propio) a uno de producci�n. Para ello, les sera explicado en detalle una arquitectura de servidores, en concreto dos, uno de control del sistema y otro en producci�n en un Cloud Hosting encargado de comunicar nuestro primer servidor y el cliente o usuario. Con mas detalle se har� incapie en la seccion [Citar seccion]

\vspace{5mm}

Por lo tanto, conseguiremos algo que no se queda en un entorno local y con ello ya podemos desarrollar distintos tipos de clientes, escritorio, m�vil etc.. Adem�s como extra se les presentar� tambi�n un cliente en formato app m�vil, utilizando Ionic como Framework de desarrollo de aplicaciones h�bridas.

\vspace{5mm}

Por �ltimo, se les destacar� en desarrollos futuros [Citar secci�n] como el alcanze del proyecto tiene una gran visi�n desde el perfeccionamiento del HardWare usado y las interconexiones de los sockets para usar mas features que la librer�a Jhonny-Five aporta.


\paragraph{Palabras clave:} Dom�tica, Arquitectura de Servidores, Conexi�n Real Time entre servidores, Node.js, Arduino, Jhonny-five, API REST.

\chapter{Abstract}

In this project...

\paragraph{Keywords:} Domotics, Server Architecture, API REST, Real time server connections, Node.js, Arduino, Jhonny-five, API REST.